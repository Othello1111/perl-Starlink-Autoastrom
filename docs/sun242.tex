\documentclass[twoside,11pt,nolof]{starlink}

% ? Specify used packages
% ? End of specify used packages

% -----------------------------------------------------------------------------
% ? Document identification
\stardoccategory    {Starlink User Note}
\stardocinitials    {SUN}
\stardocsource      {sun242.1}
\stardocnumber      {242.1}
\stardocauthors     {Norman Gray \& Brad Cavanagh}
\stardocdate        {2006 April 19}
\stardoctitle       {Autoastrometry for Mosaics}
\stardocversion     {1.5}
\stardocmanual      {User's Guide}
\stardocabstract{
We describe the \autoastrom\ package, which provides an interface to the
\ASTROM\ application. This creates a semi-automatic route to doing astrometry
on CCD images.
}

% ? End of document identification
% -----------------------------------------------------------------------------
% ? Document-specific \providecommand or \newenvironment commands.

% Also define the html equivalents.

% degrees symbol
\providecommand{\dgs}{\hbox{$^\circ$}}

% arcminute symbol
\providecommand{\arcm}{\hbox{$^\prime$}}

% arcsec symbol
\providecommand{\arcsec}{\arcm\hskip -0.1em\arcm}

% hours symbol
\providecommand{\hr}{\hbox{$^{\rm h}$}}

% minutes symbol
\providecommand{\mn}{\hbox{$^{\rm m}$}}

% seconds symbol
\providecommand{\scn}{\hbox{$^{\rm s}$}}

% decimal-minutes symbol
\providecommand{\um}{\hskip-0.3em\hbox{$^{\rm m}$}\hskip-0.08em}

% decimal-degree symbol
\providecommand{\udeg}{\hskip-0.3em\dgs\hskip-0.08em}

% decimal-second symbol
\providecommand{\us}{\hskip-0.27em\hbox{$^{\rm s}$}\hskip-0.06em}

% decimal-arcminute symbol
\providecommand{\uarcm}{\hskip-0.28em\arcm\hskip-0.04em}

% decimal-arcsecond symbol
\providecommand{\uarcs}{\hskip-0.27em\arcsec\hskip-0.02em}

% centre an asterisk
\providecommand{\lsk}{\raisebox{-0.4ex}{\rm *}}

% conditional text
\providecommand{\latexelsehtml}[2]{#1}

\hyphenation{which-ever}

% Lines for breaking up Appendices A and B.
\providecommand{\jrule}{\noindent\rule{\textwidth}{0.45mm}}
\providecommand{\krule}{\vspace*{-1.5ex}
                    \item [\rm \rule{\textwidth}{0.15mm}]}

% Frame attributes fount.  Need to find a way for these to stand out.
% San serif doesn't work by default.  Also without the \rm the
% san serif continues after \att hyperlinks.  Not sure why.  Why
% this was fixed by say having extra braces, the \sstattribute
% then used roman fount for its headings.  The current lash up
% appears to work, but needs further investigation or a TeX wizard.
\providecommand{\att}[1]{\textsf{#1}}

\providecommand{\htmlattref}[2]{\htmlref{\att{#1}}{#2}\rm}

% Shorthands for hypertext links.
% -------------------------------
\providecommand{\AGIref}{\xref{AGI}{sun48}{}}
\providecommand{\ARDref}{\xref{ARD}{sun183}{}}
\providecommand{\AST}{\xref{{\footnotesize AST}}{sun210}{}}
\providecommand{\ASTERIXref}{\xref{{\footnotesize ASTERIX}}{sun98}{}}
\providecommand{\ASTROM}{{\footnotesize ASTROM}\normalsize}
\providecommand{\ASTROMref}{\xref{{\footnotesize ASTROM}}{sun5}{}}
\providecommand{\autoastrom}{\texttt{autoastrom}}
\providecommand{\CCDPACKref}{\xref{{\footnotesize CCDPACK}}{sun139}{}}
\providecommand{\CGSDRref}{\xref{{\footnotesize CGS4DR}}{sun27}{}}
\providecommand{\CONVERT}{{\footnotesize CONVERT}\normalsize}
\providecommand{\CONVERTref}{\xref{{\footnotesize CONVERT}}{sun55}{}}
\providecommand{\CPAN}{\htmladdnormallink{\footnotesize CPAN}{http://www.cpan.org/}}
\providecommand{\CURSA}{{\footnotesize CURSA}\normalsize}
\providecommand{\CURSAref}{\xref{{\footnotesize CURSA}}{sun190}{}}
\providecommand{\ECHOMOPref}{\xref{{\footnotesize ECHOMOP}}{sun152}{}}
\providecommand{\ESPref}{\xref{{\footnotesize ESP}}{sun180}{}}
\providecommand{\EXTRACTORref}{\xref{{\footnotesize EXTRACTOR}}{sun226}{}}
\providecommand{\FIGARO}{{\footnotesize FIGARO}\normalsize}
\providecommand{\Figaroref}{\xref{{\footnotesize FIGARO}}{sun86}{}}
\providecommand{\FINDOFFref}{\xref{{\footnotesize FINDOFF}}{sun139}{}}
\providecommand{\FITSref}{\htmladdnormallink{FITS}{http://fits.gsfc.nasa.gov/}}
\providecommand{\GAIA}{{\footnotesize GAIA}\normalsize}
\providecommand{\GAIAref}{\xref{GAIA}{sun214}{}}
\providecommand{\GWM}{{\footnotesize GWM}\normalsize}
\providecommand{\GWMref}{\xref{GWM}{sun130}{}}
\providecommand{\HDSref}{\xref{HDS}{sun92}{}~}
\providecommand{\HDSTRACEref}{\xref{{\footnotesize HDSTRACE}}{sun102}{}}
\providecommand{\ICL}{{\footnotesize ICL}\normalsize}
\providecommand{\ICLref}{\xref{{\footnotesize ICL}}{sg5}{}}
\providecommand{\IRCAMPACKref}{\xref{{\footnotesize IRCAMPACK}}{sun177}{}}
\providecommand{\IRAF}{\footnotesize{IRAF}\normalsize}
\providecommand{\IRAFref}{\htmladdnormallink{\IRAF}{http://iraf.noao.edu/iraf-homepage.html}}
\providecommand{\IRASref}{\xref{{\footnotesize IRAS90}}{sun163}{}}
\providecommand{\JCMTDRref}{\xref{{\footnotesize JCMTDR}}{sun132}{}}
\providecommand{\KAPLIBS}{{\footnotesize KAPLIBS}\normalsize}
\providecommand{\KAPPA}{{\footnotesize KAPPA}\normalsize}
\providecommand{\KAPRHref}{\xref{{\footnotesize KAPRH}}{sun239}{}}
\providecommand{\NDFref}[1]{\htmlref{#1}{ap:NDFformat}~}
\providecommand{\NDFextref}[1]{\xref{#1}{sun33}{}}
\providecommand{\NDFPACK}{{\footnotesize NDFPACK}\normalsize}
\providecommand{\PDAref}{\xref{PDA}{sun194}{}}
\providecommand{\PGPLOT}{{\footnotesize PGPLOT}\normalsize}
\providecommand{\PGPLOTref}{\htmladdnormallink{\PGPLOT}{http://astro.caltech.edu/\~{}tjp/pgplot/}}
\providecommand{\PHOTOMref}{\xref{{\footnotesize PHOTOM}}{sun45}{}}
\providecommand{\PISAref}{\xref{{\footnotesize PISA}}{sun109}{}}
\providecommand{\PONGOref}{\xref{{\footnotesize PONGO}}{sun137}{}}
\providecommand{\PSMERGEref}{\xref{{\footnotesize PSMERGE}}{sun164}{}}
\providecommand{\TSPref}{\xref{{\footnotesize TSP}}{sun66}{}}
\providecommand{\POLPACKref}{\xref{{\footnotesize POLPACK}}{sun223}{}}
\providecommand{\SExtractor}{\xref{{\footnotesize SExtractor}}{sun226}{}}
\providecommand{\SkyCat}{\htmladdnormallink{SkyCat}{http://archive.eso.org/skycat/}}
\providecommand{\SkyCatFAQ}{\htmladdnormallink{SkyCat FAQ}{http://archive.eso.org/skycat/skycat-faq.html}}
\providecommand{\STLref}{\xref{{\footnotesize STL}}{sun190}{STLREF}}
\providecommand{\TWODSPECref}{\xref{{\footnotesize TWODSPEC}}{sun16}{}}


% ? End of document specific commands
%------------------------------------------------------------------------------
%  Title Page.
%  ===========
\begin{document}
\scfrontmatter

\section{\xlabel{se_intro}Introduction\label{se:intro}}

\textbf{THIS IS BETA SOFTWARE. Some features work only partially or
  unreliably. It does not have all the functionality of a production
  release. The interface may well change.}

Starlink's \ASTROMref\ application provides powerful astrometry facilities, for
analysing astronomical images; it is, however, rather cumbersome to use. As
described in \xref{SUN/5}{sun5}{}, \ASTROM\ can:

\begin{itemize}
\item obtain the plate centre and plate scale of a CCD image, by comparing
  plate positions of objects with those in a reference catalogue, and fitting
  with both four- and six-component distortion models;
\item with enough data, obtain cubic distortion and plate tilt, using seven-
  to nine-component models;
\item if enough information is available, it will do the reductions in
  observed place, correcting for atmospheric refraction.
\end{itemize}

\autoastrom\ provides a shell around \ASTROM\ so that, as well
as the core astrometric facilities of \ASTROM, \autoastrom\
will:

\begin{itemize}

\item work on a CCD image provided as an NDF, as long as it has at
least rough astrometry (plate centre and scale);

\item automatically download appropriate reference catalogue information
from catalogue servers supported by the SkyCat library;

\item insert the astrometric results into the original NDF, as well as
a WCS component; \item alternatively or additionally make the
astrometry available as a set of FITS-WCS header cards.  \end{itemize}

\section{\xlabel{se_usage}Usage\label{se:usage}}

Summary:
\begin{terminalv}
autoastrom [options...] <NDF>
\end{terminalv}

The options available are described in Table \ref{tab:options}. String or
integer values are given in the form \texttt{--catalogue=usno@eso}; those
marked \texttt{switch}, such as \texttt{--insert}, can be specified as either
\texttt{--insert} (meaning true) or \texttt{--noinsert} (false).

Those options which take values can have those values specified as either
\texttt{--option=value} or \texttt{--option value}. The latter form is rather
unfortunate (but currently unavoidable), since it means that if you forget to
give a value, it gobbles the next option on the line as its value. The options
of type \texttt{switch} below take no value; all of the others do.

\begin{table}
\begin{center}
\begin{tabular}{l|l|p{8cm}}
Option & Type & Description \\ \hline
\texttt{--bestfitlog} & string & File to hold information about the best fit \\
\texttt{--catalogue} & string & SkyCat name of online catalogue to use \\
\texttt{--ccdcatalogue} & string & Catalogue to use in place of extracting
objects from CCD \\
\texttt{--defects} & string & Remove CCD defects \\
\texttt{--detectedcatalogue} & string & Dump catalogue of detected objects
with updated WCS information to this file \\
\texttt{--help} & switch & Print usage information and exit \\
\texttt{--insert} & switch & Insert WCS into NDF at end? \\
\texttt{--iterrms\_abs} & real & Absolute RMS level to reach to stop iterations
\\
\texttt{--iterrms\_diff} & real & Differential RMS level to reach to stop
iterations \\

\texttt{--keepfits} & string & Retain WCS in this file (default if empty) \\
\texttt{--keeptemps} & switch & Keep temporary files? \\
\texttt{--match} & string & Choose matching algorithm \\
\texttt{--matchcatalogue} & string & Dump catalogue of matches to this file \\
\texttt{--maxfit} & integer & Maximum number of fit parameter to use \\
\texttt{--maxiter} & integer & Maximum number of iterations to perform \\
\texttt{--maxobj\_corr} & integer & Maximum number of objects to use for
correlation \\
\texttt{--maxobj\_image} & integer & Maximum number of objects to use from CCD
\\
\texttt{--maxobj\_query} & integer & Maximum number of objects to use from
SkyCat query \\
\texttt{--messages} & switch & Show Starlink application messages? \\
\texttt{--obsdata} & string & Provide observation information, including WCS
information \\
\texttt{--skycatconfig} & string & SkyCat configuration file \\
\texttt{--skycatcatalogue\_in} & string & Catalogue to use in place of SkyCat
query \\
\texttt{--skycatcatalogue\_out} & string & Dump SkyCat query result catalogue
to this file \\
\texttt{--temp} & string & Name of temporary directory \\
\texttt{--timeout} & integer & Monolith timeout \\
\texttt{--verbose} & switch & Verbosity \\
\texttt{--version} & switch & Print version number and exit \\

\end{tabular}
\end{center}
\caption{\label{tab:options}\autoastrom\ options. Further details are
  in Section \ref{sb:options}}
\end{table}

\subsection{\xlabel{sb_specifyingpositions}Specifying
  positions\label{sb:specifyingpositions}}

To work, \autoastrom\ requires some initial estimate of the centre of the CCD
frame. You can provide this with a WCS component in the NDF, or with a FITS
(NDF-) extension, or with approximate astrometry given on the command line.
By default, \autoastrom\ searches for a WCS component, then a FITS extension,
and fails if it finds neither.  Because \autoastrom\ ultimately sits on top of
the \AST\ library, it is able to make sense of a large variety of embedded
FITS information, and get its initial astrometry from such pointing
information.

  You control this process using the \texttt{--obsdata} option.

  If you wish to insert an initial approximate WCS component in the NDF, you
  can best do this using \GAIA.

  The initial calibration does not have to be particularly accurate; it need
  only be accurate enough that \autoastrom\ is able to make a query to a
  catalogue server that will return a set of catalogue objects that has a
  substantial overlap with the imaged area of sky.  The query covers the
  region of sky which maps to the four corners of the image, based on the
  initial astrometry, plus a small extra margin.

  The default image scale is 1 arcsec per pixel.  You will need to specify the
  scale if the actual scale is significantly different from this, as errors in
  this scale result in substantial errors in the patch of sky requested from
  the server.  It is better to choose too small a scale than too large -- that
  way, the area which \autoastrom\ searches for will tend to lie within the
  observed area, rather than greatly overlap it.

  \autoastrom\ assumes that north is along the \texttt{y}-axis.  The matching
  algorithms are generally insensitive to this angle.  However, the only case
  where you need to specify this is if the CCD is significantly longer than it
  is wide, since in this case a wrong orientation would cause \autoastrom\ to
  search for a patch of sky which was a different shape from the patch covered
  by the CCD.  Similarly, you should specify any image inversion if you know
  it, but it is not always necessary.

\subsection{\xlabel{sb_iterations}Iterations\label{sb:iterations}}

\autoastrom\ comes to its solution iteratively. After acquiring both
catalogues -- the SkyCat catalogue and the detected objects catalogue -- it
enters an iteration loop, starting off by matching objects between the two
catalogues. It then calculates an astrometric solution, and if certain
criteria have been met, the calculated solution is used. Otherwise, the
iteration loop is entered again. This has the effect that more objects are
usually matched between the two catalogues after an improved astrometric
solution has been found, which leads to a more accurate solution.

Iterations can be halted in one of three ways: by reaching an absolute RMS
level in the astrometric solution; by reaching a limit in the difference
between the current astrometric solution's RMS and the prior solution's; and
by reaching a maximum number of iterations. By default \autoastrom\ is set to
halt after ten iterations. For more information on modifying these values, see
Sections \ref{sb:options:iterrms_abs}, \ref{sb:options:iterrms_diff}, and
\ref{sb:options:maxiter}.

\subsection{\xlabel{sb_results}Results\label{sb:results}}

Results can be inserted into the input NDF (see
Section~\ref{sb:options:insert}), or else retained as a standalone FITS-WCS
file (see Section~\ref{sb:options:keepfits}).  You can choose both, either, or
even choose neither and throw the results away (strange person!).

The FITS files which \autoastrom\ produces are in fact generated by \ASTROM.
As discussed in \xref{SUN/5}{sun5}{}, these employ a slight extension to the FITS-WCS
standards, since they necessarily include distortion information which has
been deferred until Paper~III \cite{fitswcs3}.  Nonetheless, they are
perfectly conformant.  If, however, \autoastrom\ had to be configured with
support for an old version of \AST, then it might have had to generate FITS
headers which conformed to the drafts of paper~II, and thus do not conform to
the final version -- such FITS files should not be used outside of this
application, and it would be best to update your version of AST to a current
one, and reinstall \autoastrom.

The calculated astrometry is inserted into the input NDF unless you suppress
it with the \texttt{--noinsert} option.  If you need to insert this by hand,
into this or another NDF, you can do so with \KAPPA's wcscopy command (you may
need to initialise the \CONVERT\ package, so that the conversions between FITS
and NDF will happen automatically).

For further details, see the FITS standard \cite{nost-100-2} and the three
published FITS-WCS papers, \cite[FITS WCS Paper I]{fitswcs1}, \cite[FITS WCS
Paper II]{fitswcs2} and \cite[FITS WCS Paper III]{fitswcs3}; the web pages
which cover the FITS-WCS process are at
\url{http://www.cv.nrao.edu/fits/documents/wcs/wcs.html}.

\subsection{\xlabel{sb_webproxies}Web proxies\label{sb:webproxies}}

In order to perform catalogue lookups, \autoastrom\ needs to make HTTP (Web)
queries.  If your site enforces the use of proxies to get access to the web,
these lookups will fail, and \autoastrom\ will simply wait uselessly until its
queries time out.

You can tell \autoastrom\ how to use the proxies by using the standard
\texttt{HTTP\_PROXY} environment variable, which you set through something
like

\begin{terminalv}
csh% setenv HTTP_PROXY http://wwwcache.example.ac.uk:8080/
\end{terminalv}
or
\begin{terminalv}
sh% HTTP_PROXY=http://www.example.ac.uk:8080/; export HTTP_PROXY
\end{terminalv}
depending on whether you use a csh-like (tcsh) or sh-like (bash, zsh) shell.

Your local computing service can advise you on the appropriate value for this
environment variable, or you can examine the appropriate part of the
configuration of your web browser.

\subsection{\xlabel{sb_options}Options\label{sb:options}}

The options are either switches, or have a string or integer argument.

The verbosity option is an example of a switch: it can be given as
\texttt{--verbose} to make the program verbose, or \texttt{--noverbose} to
make it quiet.

Some options take suboptions.  For example, the \texttt{--obsdata} option
might be given as

\begin{terminalv}
--obsdata=ra=12:34:56,dec=-10:0:0
\end{terminalv}

where the comma-separated list of suboptions is given as the
\texttt{--obsdata} option argument.

There are no optional arguments: if an option is documented as having an
argument (that is, if it is not a switch), an argument must be provided.  The
defaults documented below apply only if the option is not supplied.

\subsubsection{\xlabel{sb_options_bestfitlog}--bestfitlog\label{sb:options:bestfitlog}}

When \ASTROM\ finds a solution, information about the best fit is written to
this file. This file is a textfile containing the number of terms, the central
right ascension and declination, the average pixel RMS, the distortion factor
(when applicable) and the FITS file containing the WCS headers for all of the
fits found. For the best fit, this file also lists the number of stars used in
the fit, the mean plate scale in arcseconds, and the RMS in fitted $r$, $x$,
and $y$.

Type: string; default: none.

\subsubsection{\xlabel{sb_options_catalogue}--catalogue\label{sb:options:catalogue}}

This is the name of the online catalogue to use to find reference stars.  The
argument should be one of the catalogue names recognised by \SkyCat.  The list
of available catalogues depends on the configuration file used (see
Section~\ref{sb:options:skycatconfig}); the `\texttt{short\_name}' entries in
the file are the allowable values of this option.

Type: string; default (presently): `usno@eso'

\subsubsection{\xlabel{sb_options_ccdcatalogue}--ccdcatalogue\label{sb:options:ccdcatalogue}}

Specifies a pre-existing catalogue of objects in the CCD frame. The format is
that produced as output by \SExtractor\ when \texttt{catalog\_type} is set to
\texttt{ASCII\_HEAD}. The file starts with a sequence of lines of the form
\begin{terminalv}
# <column> <colname> <optional comment>
\end{terminalv}

The \textit{colnum} numbers run from 1 to the number of columns; the
\textit{colname} is one of a set of column named defined in the \SExtractor\
documentation; and \textit{optional comment} is some (ignored) annotation.
The data immediately follows, in rows with the given number of columns in
them.  In each of these rows, any text following a \texttt{\#} is ignored, and
may be used for other annotation.

You control which columns SExtractor generates by specifying them in the
SExtractor \texttt{.param} file.  The catalogue must have all of the fields
\texttt{NUMBER}, \texttt{FLUX\_ISO}, \texttt{X\_IMAGE}, \texttt{Y\_IMAGE},
\texttt{A\_IMAGE}, \texttt{B\_IMAGE}, \texttt{X2\_IMAGE}, \texttt{Y2\_IMAGE},
\texttt{ERRX2\_IMAGE}, \texttt{ERRY2\_IMAGE}, \texttt{ISOAREA\_IMAGE}, of which
\texttt{X2\_IMAGE}, \texttt{Y2\_IMAGE}, \texttt{A\_IMAGE}, \texttt{B\_IMAGE},
\texttt{ERRX2\_IMAGE}, \texttt{ERRY2\_IMAGE} are not generated by default.

For further details about SExtractor, and details of these required columns,
see Section~\ref{se:auxiliary}.

If you include the option \texttt{--keeptemps} to avoid deleting the
temporary work directory, you will be able to scavenge the extractor output
from there if this is useful to you.

Type: string; default: none

\subsubsection{\xlabel{sb_options_defects}--defects\label{sb:options:defects}}

This option is currently unsupported but will be reinstated at some point in
the future. The following documentation is valid for older versions of
\autoastrom.

If this option is present, then when the application extracts the list of
objects from the CCD, it will try to remove CCD blemishes.  The algorithm is
currently rather crude, and will likely change in future.

Some of the `objects' detected by EXTRACTOR are in fact CCD defects, or
readout errors, or the like.  These are very bright, so they can confuse a
matching program which examines only or preferentially the brightest objects.

By default, \autoastrom\ will warn of the existence of anything it thinks is a
defect, based on a plausible heuristic, and you can control this using this
option.  The options takes a list of keywords, which it processes as shown in
Table~\ref{tab:defects}.

\begin{table}
\begin{center}
\begin{tabular}{l|p{10cm}}
Keyword & Description \\ \hline
ignore & Completely ignore defects. \\
warn & Warn about possible defects, but do nothing further. This is the
default. \\
remove & Remove any suspected defects from the catalogue of CCD objects. \\
badness & Provide the threshold for defect removal and warnings. Any objects
with a badness greater than the value specified here are noted or removed. The
default threshold is 1.
\end{tabular}
\end{center}
\caption{\label{tab:defects}Keywords for \texttt{--defects} option}
\end{table}

The heuristic works by assigning a 'badness' to each object on the CCD.
Objects with a position variance, $<x^2>$ or $<y^2>$, smaller than one pixel,
and objects whose flux density (counts/pixel) is significantly higher than the
average, are given high scores.  One can observe that line and point defects
score high with this, with a badness greater than 1, but this is not
completely reliable.

We can afford a few uncaught defects, and we can afford to discard a few real,
but very small, sources, since the match algorithms will generally simply
ignore these.  What we need to avoid is a CCD catalogue which is dominated by
bright sources which have no counterpart on the sky.

We emphasise that this defect-removal algorithm is not particularly
sophisticated -- if the object-extraction is producing spurious objects, you
may need to mask the defects out by hand, using GAIA or similar.

If you have any observations on the reliability, or indeed usefulness, of this
option, the author would be interested to receive them.

This option is not always necessary when used with the default FINDOFF
matching algorithm, but it is vital when used with the `match' matching
algorithm (see Section~\ref{sb:options:match}), since that algorithm uses only
the brightest objects detected.

Type: string; default: \texttt{--defects=warn}

\subsubsection{\xlabel{sb_options_detectedcatalogue}--detectedcatalogue\label{sb:options:detectedcatalogue}}

Specifies a file which receives a dump of the objects detected in the NDF,
with their astrometric positions corrected. The file is in the Cluster format
with thirteen columns: a zero in the first column, followed by the item's ID,
the right ascension and declination in space-separated format, the x and y
position of the source on the CCD, an instrumental magnitude, the error in the
instrumental magnitude, and extraction flags.

Type: string; default: none.

\subsubsection{\xlabel{sb_options_help}--help\label{sb:options:help}}

Print usage information, and defaults, and exit.

Type: switch; default: false.

\subsubsection{\xlabel{sb_options_insert}--insert\label{sb:options:insert}}

If true, the final fit to the astrometry is inserted into the input NDF as an
AST WCS component.  If false (\texttt{--noinsert}), the insertion is not done,
so presumably you have decided to handle the WCS information in a different
way, and have set the \texttt{--keepfits} option to retain the WCS
information.

Type: switch; default: true.

\subsubsection{\xlabel{sb_options_iterrms_abs}--iterrms\_abs\label{sb:options:iterrms_abs}}

When the absolute RMS of a fit reaches this value, iterations will cease and
the current WCS will be used. For more information on \autoastrom's iteration
process, see Section~\ref{sb:iterations}.

Type: real; default; none.

\subsubsection{\xlabel{sb_options_iterrms_diff}--iterrms\_diff\label{sb:options:iterrms_diff}}

When the difference between the current RMS and the previous fit's RMS reaches
this value, iterations will cease and the current WCS will be used. For more
information on \autoastrom's iteration process, see
Section~\ref{sb:iterations}.

Type: real; default; none.

\subsubsection{\xlabel{sb_options_keepfits}--keepfits\label{sb:options:keepfits}}

The normal mode of operation is for the script to insert the final astrometry
into the input NDF, as an AST WCS component (see
Section~\ref{sb:options:insert}).  Instead, or additionally, you can save this
information as a FITS-WCS file, named by this option.

Type: switch; default: false.

\subsubsection{\xlabel{sb_options_keeptemps}--keeptemps\label{sb:options:keeptemps}}

If true, temporary files will not be deleted at the end of processing.  This
is useful for debugging, both at the low level when there is something wrong
with the script itself, but also when you wish to examine the sequence of
calls to FINDOFF and \ASTROM, or examine the \ASTROM\ input and output
files.

All the temporary files are created in a temporary directory which is
reported during the script's execution.

Type: switch; default: false

\subsubsection{\xlabel{sb_options_man}--man\label{sb:options:man}}

If true, print a man page and exit.

Type: switch; default: false

\subsubsection{\xlabel{sb_options_match}--match\label{sb:options:match}}

Specifies an alternative matching algorithm.  The default algorithm is
`match', which is an implementation of the FOCAS matching algorithm
\cite{valdes95} by Michael W Richmond \cite{match-home}.  A version of this is
distributed with \autoastrom, but it should be compatible with any later
version.

Alternatively, you can use the FINDOFF application, which is part of
\CCDPACKref.  This does work, but is rather slow, and is completely thrown by
a CCD image with unequal X and Y scales, making it essentially unusable if the
image NDF does not have some astrometry already, or if you try to specify a
position with \texttt{--obsdata} (see Section~\ref{sb:options:obsdata}).
Since it has different characteristics, though, it may work in some
circumstances where `match' does not.  Again, the author of \autoastrom\ would
welcome feedback on this point.

\autoastrom\ supports `match' through the more general plugin mechanism
described in Section~\ref{sb:plugins}.

Type: string; default: `match'.

\subsubsection{\xlabel{sb_options_matchcatalogue}--matchcatalogue\label{sb:options:matchcatalogue}}

Specifies a file which receives a dump of the set of positions matched by the
matching process.  The file is formatted like a \SExtractor\ output file, with
five columns, containing (1) running object number, (2 and 3) RA and Dec of
the source on the sky, and (4 and 5) \textbf{x} and \textbf{y} positions of
the source on the CCD.

Type: string; default: absent -- no file written by default.

\subsubsection{\xlabel{sb_options_maxfig}--maxfit\label{sb:options:maxfit}}

The maximum order fit to use. If not set, then 9 will be used.

Type: integer; default: 9.

\subsubsection{\xlabel{sb_options_maxiter}--maxiter\label{sb:options:maxiter}}

The maximum number of iterations to perform. For more information on
\autoastrom's iteration process, see Section~\ref{sb:iterations}.

Type: integer; default: 10.

\subsubsection{\xlabel{sb_options_maxobj_corr}--maxobj\_corr\label{sb:options:maxobj_corr}}

The maximum number of objects to use from either catalogue to perform
correlation on. This will take the brightest objects in the catalogues.

Type: integer; default: 500.

\subsubsection{\xlabel{sb_options_maxobj_image}--maxobj\_image\label{sb:options:maxobj_image}}

The maximum number of objects to use from the image catalogue. This will take
the brightest objects in the catalogue.

Type: integer; default: 500;

\subsubsection{\xlabel{sb_options_maxobj_query}--maxobj\_query\label{sb:options:maxobj_query}}

The maximum number of objects to fetch from the catalogue server. The default
is generous, and it's unlikely you'd need to change this.

Type: integer: default: 500

\subsubsection{\xlabel{sb_options_messages}--messages\label{sb:options:messages}}

If true, the script will pass on messages from the Starlink applications which
the script uses.  These might be reassuring, but if you don't like the
chatter, they can be suppressed with \texttt{--nomessages}.

Type: switch; default: true

\subsubsection{\xlabel{sb_options_obsdata}--obsdata\label{sb:options:obsdata}}

Specifies a source for the observation data, including WCS information.  The
keyword has the multiple role of specifying the source of WCS information,
supplying approximate WCS information directly, and supplying the observation
data (time, observatory, temperature andpressure, and colour) which \ASTROM\
needs if it is to attempt one of its higher-order fits.  This information is
only needed if it is missing from the NDF or any FITS extension it
incorporates, and if you wish to attempt the slight increase in accuracy of
the more elaborate fits.

Do not specify a keyword more than once with distinct values.

The value of the \texttt{--obsdata} option can specify approximate astrometry
through a comma-separated list of \texttt{key=value} pairs.  For example, to
supply a value for the centre of a CCD, overriding any astrometry in the file,
you might write
\begin{terminalv}
 --obsdata=ra=14:24:00,dec=-12.34
\end{terminalv}
This value indicates the centre of the CCD even when this is not the centre of
distortion, which would be the case when, for example, the CCD represents an
off-centre section of a Schmidt plate.

The program will take its astrometry from only one of the possible sources
(see \texttt{source} below) -- so that it will not, for example, take a
position from the \texttt{obsdata} option and a plate scale from the file's
own astrometry.  By default, a position given here overrides any other WCS
information; thus to supply a position as a backup, in case there is
\textbf{no} astrometry elsewhere, you could write
\begin{terminalv}
--obsdata=source=AST:FITS:USER,ra=14:24:00,dec=-12.34
\end{terminalv}
The available astrometry keywords are in Table~\ref{tab:obsdata}.

\begin{table}
\begin{center}
\begin{tabular}{l|p{10cm}}
Keyword & Description \\ \hline
source & This is a colon-separated list of sources of WCS information. The
values may be `AST', indicating that the information should come from the AST
WCS component of the NDF, or `FITS', indicating that it should come from any
FITS extension there, or `USER', indicating that the following
\texttt{obsdata} keywords should be used. The default is
\texttt{source=USER:AST:FITS}, so that any WCS information given in the
\texttt{obsdata} keyword has precedence. The keywords are not
case-sensitive. If the program gets through this list without finding any WCS
information, it exits with an error.\\
ra & Right ascension of the centre of the pixel grid, in colon-separated HMS
or decimal degrees. Required.\\
dec & Declination of the centre of the pixel grid, in colon-separated DMS or
decimal degrees. Required.\\
angle & Position angle of the pixel grid. This is the rotation, in degrees
anti-clockwise, of the declination or latitude axis with respect to the 2-axis
of the data array. Default: 0.\\
scale & The plate scale, in units of arcsec/pixel. Default: 1.\\
invert & If true (that is, specified as \texttt{invert=1} or just
\texttt{invert}), the axes are inverted; otherwise (\texttt{invert=0} or
\texttt{noinvert}) the axes are unflipped. Default: 0.
\end{tabular}
\end{center}
\caption{\label{tab:obsdata}Keywords for \texttt{--obsdata} option}
\end{table}

The position angle is the same as the AIPS \texttt{CROTA2}
convention. Specifically (and by definition) these definitions map to the
\texttt{CDn\_n} conventions of \cite{fitswcs2} through
\begin{align*}
\texttt{CD1\_1} & = \pm s \cos( \phi ) \\
\texttt{CD1\_2} & = \pm s \sin( \phi ) \\
\texttt{CD2\_1} & = - s \sin( \phi ) \\
\texttt{CD2\_2} & = s \cos( \phi ),
\end{align*}
where \textit{s} is \texttt{scale}/3600 (ie, the scale in unites of
degrees/pixel), $\phi$ is the position angle \texttt{pa}, and $\pm$ is $+$
when \texttt{invert} is false, and $-$ when \texttt{invert} is true.

Additionally, there are keywords which specify observation data: time, station
coordinates, meteorological data and colour.  These are described in
Table~\ref{tab:datetime}.  For fuller details on these, see \ASTROMref.  Note
that the \texttt{source} keyword discussed above does not affect the priority
of these keywords: if they are specified, they override any values obtained
from the NDF or its FITS extensions.

\begin{table}
\begin{center}
\begin{tabular}{l|p{10cm}}
Keyword & Description \\ \hline
time & An observation time, given as a Julian epoch (format \texttt{r}), or a
local sidereal time (format \texttt{i:i}) or UT (\texttt{i:i:i:i:r} specifying
four-digit year, month, day, hours, and minutes). \\
obs & An observation station, given either as one of the SLALIB observatory
codes; or else in the format \texttt{i:r:i:r[:r]}, giving longitude, latitude
and optional height. Longitudes are east longitudes -- west longitudes may be
given either as minus degrees or longitudes greater than 180. \\
met & Temperature and pressure at the telescope, in degrees Kelvin and
millibars. The defaults are 278K and a pressure computed from the observatory
height. Format \texttt{r[:r]}. \\
col & The effective colour of the observations, as a wavelength in
nanometres. The default is 500nm.
\end{tabular}
\end{center}
\caption{\label{tab:datetime}Observation data keywords. In these
  specifications, \texttt{i} represents an integer, \texttt{r} a real,
  optional entries are in \texttt{[...]}, and the separator \texttt{:} may be
  either a colon or whitespace.}
\end{table}

Type: string; defaults: see above.

\subsubsection{\xlabel{sb_options_skycatconfig}--skycatconfig\label{sb:options:skycatconfig}}

The SkyCat library has a default configuration file, which you can examine at
\url{http://archive.eso.org/skycat/skycat2.0.cfg}.  For some applications,
it may be appropriate to use an alternative configuration file.  If so, you
can point to it using this option.  For more information on SkyCat,
catalogues, and the configuration file, see the \SkyCatFAQ.

Type: string; default: built-in defaults.

\subsubsection{\xlabel{sb_options_skycatcatalogue_in}--skycatcatalogue\_in\label{sb:options:skycatcatalogue_in}}

Specifies a pre-existing catalogue of objects as retrieved from SkyCat. The
format of this catalogue must be the Cluster format as described in
Section~\ref{sb:options:detectedcatalogue}.

Type: string; default: none.

\subsubsection{\xlabel{sb_options_skycatcatalogue_out}--skycatcatalogue\_out\label{sb:options:skycatalogue_out}}

After the SkyCat query has been performed, the catalogue returned can be saved
to disk in this file. The format of this catalogue will be the Cluster format
as described in Section~\ref{sb:options:detectedcatalogue}.

If used in conjunction with the \texttt{--skycatcatalogue\_in} option (see
Section~\ref{sb:options:skycatcatalogue_in}), this can be used as a poor-man's
cache, avoiding potentially timely SkyCat queries.

Type: string; default: none.

\subsubsection{\xlabel{sb_options_temp}--temp\label{sb:options:temp}}

This option, if present, specifies the directory to be used for temporary
files.

This has a double function: firstly and most straightforwardly, it indicates
where temporary files should be created if, for some reason, the default
location is unsuitable; secondly, if the script finds there a file (from a
previous run) which it was about to generate, it does not regenerate it, but
instead simply reuses it.  The first function may be useful in some
circumstances; the second is primarily a debugging facility, is subtle, and
its use is discouraged.

Type: string; default: generated.

\subsubsection{\xlabel{sb_options_timeout}--timeout\label{sb:options:timeout}}

If present, this specifies how long, in seconds, the script should wait for
its slave applications to complete.  The default is three minutes.  In some
circumstances -- such as if you have a huge number of stars to match up --
this might be too short, so that the script loses patience before the slave
application has finished, and produces a rather messy error message.  In this
circumstance, you can increase the timeout here.

Type: integer; default: 180.

\subsubsection{\xlabel{sb_options_verbose}--verbose\label{sb:options:verbose}}

If present, the script produces a good deal of chatter to the standard error;
if false (\texttt{--noverbose}), this is suppressed.

Type: switch; default: false.

\subsubsection{\xlabel{sb_options_version}--version\label{sb:options:version}}

If present, write the version number ot the standard output and exit.

Type: switch; default: false.

\section{\xlabel{se_auxiliary}Auxiliary Software\label{se:auxiliary}}

The \autoastrom\ script relies primarily on the \texttt{Starlink::Autoastrom}
Perl module, which relies on a number of other Perl modules which, in turn,
rely on a number of Starlink software tools. All of these tools should be
up-to-date if you are using a Starlink distribution as obtained from the
Starlink CVS repository.

The pattern matching facilities which \autoastrom\ relies come from the
FINDOFF application within \CCDPACKref. There are further details in
\xref{SUN/139}{sun139}{}. You need version 4.0-1 at least, and the program respects any setting
of the \texttt{CCDPACK\_DIR} environment variable.

\autoastrom\ requires \AST\ 1.8-1 or better. See \xref{SUN/211}{sun211}{} on information about
AST.

Object detection is performed by Extractor, which is the Starlink version of
Emmanuel Bertin's \SExtractor. Extractor is documented in \xref{SUN/226}{sun226}{}. See also
the SExtractor manual MUD/165, the home of SExtractor on the web
\cite{sextractor-web}, the distribution location \cite{sextractor-ftp}, and
the original SExtractor article \cite{sextractor-article}.

If you need to supply the inital astrometry which \autoastrom\ requires, you
can do that using \GAIA. This is documented in \xref{SUN/214}{sun214}{}, but it's very easy to
use, and you might be best to simply start it up (give the command
\texttt{gaia image\_name}, presuming you have already done the initialisation
to use Starlink software) and experiment.

\subsection{\xlabel{sb_environment}Environment\label{sb:environment}}

The environment variable \texttt{ASTROM\_DIR} points to the installed location
of \ASTROM, \texttt{CCDPACK\_DIR} points to \CCDPACKref, and
\texttt{EXTRACTOR\_DIR} points to \SExtractor. All of these environment
variables will be set up for you when you do the initialisation to use
Starlink software.

\subsection{\xlabel{sb_plugins}Plugins\label{sb:plugins}}

\autoastrom\ has, through its use of the \texttt{Astro::Correlate} Perl
module, a flexible extension mechanism for the matching of objects between
those found on the CCD and those downloaded from an online catalogue.

For further information on creating new plugins, see the documentation for the
\texttt{Astro::Correlate} Perl module, which can be found on \CPAN.

\section{\xlabel{se_future}Future developments\label{se:future}}

As a longer-term goal, it \textit{might} be possible to automate even the
discovery of the initial pointing, if there is some reasonable way of
estimating plate-centre and plate-scale information \textit{ex nihil}, from
the pattern of objects in the plate.

At present, only one CCD can be processed at a time. It would be useful if the
application could process images from multiple CCDs arranged in a mosaic.

\subsection{\xlabel{sb_restrictions}Restrictions and
  limitations\label{sb:restrictions}}

The script inherits some limitations from \ASTROM, and some from CCDPACK's
FINDOFF.

The limitations due to \ASTROM\ are as follows (quoted from \xref{SUN/5, Section~5}{sun5}{limitation}):

\begin{quote}

\ASTROM\ aims to deliver results better than 1\arcsec\ from typical Schmidt
plate measurements, and better than 0.1\arcsec\ from carefully measured JKT and
AAT plates \textit{etc}.  Astrometric specialists will, nonetheless, be aware
of a number of shortcomings, including the following:

\begin{itemize}
\item The fit is limited to a 6-coefficient linear model plus cubic distortion
  and plate tilt.  Colour effects -- arising for example from chromatic
  aberrations in the camera optics -- are not allowed for, no magnitude or
  image shape terms are included in the model, and the refraction cannot be
  adjusted automatically.

\item The zonal distortions of the reference catalogues are neglected.

\item There is no provision for the simultaneous fitting of more than one
  plate.  This prevents an extended area being modelled via overlapping
  plates, and the determination of proper motion and parallax from plates
  taken at different epochs.

\item Only rudimentary error information is produced.
\end{itemize}

Despite these limitations, which stem mainly from the need for simplicity of
use, the accuracy of the result tends in practice to be dominated by the
quality of the input data rather than by \ASTROM\ itself.

\end{quote}

Further, \ASTROM\ requires that at least 10 reference stars are available.
Since \autoastrom\ obtains its reference stars from well-stocked catalogues,
this is not a problem in practice.

The restrictions arising from FINDOFF are similarly slight, and should not be
a problem if the initial astrometry is good enough.  It's not at present
\textit{completely} clear exactly what `good enough' means.

\begin{thebibliography}{}

\bibitem{sextractor-article} Bertin, E. Sextractor: Software for source
  extraction. \textit{Astron. Astrophys. Suppl. Ser.}, 117:393, 1996.

\bibitem{sextractor-ftp} Bertin, E. Sextractor distribution [online]. 2001
  [cited 25 April 2005]. URL:
  \texttt{$<$ftp://ftp.iap.fr/pub/from\_users/bertin/sextractor/$>$}.

\bibitem{sextractor-web} Bertin, E. SExtractor program [online, cited 25 April
  2005] URL:
  $<$\url{http://terapix.iap.fr/rubrique.php?id\_rubrique=91/}$>$.

\bibitem{fitswcs2} Calabretta, M.R. \& Greisen, E.W. Representations of
  celestial coordinates in FITS (``Paper II''). \textit{Astronom. Astrophys.},
  395:1077-1122, 2002.

\bibitem{fitswcs1} Greisen, E.W. \& Calabretta, M.R.. Representations of world
  coordinates in FITS (``Paper I''). \textit{Astronom. Astrophys.},
  395:1061-1075, 2002.

\bibitem{fitswcs3} Greisen, E.W., Calabretta, M.R., Valdes, F.G., \& Allen,
  S.L. Representations of spectral coordinates in FITS (``Paper III'').
  \textit{Astronom. Astrophys.}, 446:747-771, 2006.

\bibitem{nost-100-2} Hanisch, R.J., Farris, A., Greisen, E.W., Pence, W.D.,
  Schlesinger, B.M., Teuben, P.J., Thompson, R.W., \& Warnock, A. Definition
  of the flexible image transport system (FITS). \textit{Astron. Astrophys.},
  376:359-380, September 2001.

\bibitem{match-home} Richmond, M.W. Matching lists of stars [online, cited
  April 2005] URL: $<$\url{http://spiff.rit.edu/match/}$>$

\bibitem{valdes95} Valdes, F.G., Campusano, L.E., Velasquez, J.D., \& Stetson,
  P.B. FOCAS automatic catalog matching algorithms. \textit{Publications of
    the Astronomical Society of the Pacific}, 107:1119, November 1995.

\end{thebibliography}

\end{document}

%%% Local Variables:
%%% mode: latex
%%% TeX-master: t
%%% End:
